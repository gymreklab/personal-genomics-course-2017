\documentclass[12pt]{article}

% Imports
\usepackage{hyperref}
\usepackage[margin=0.5in]{geometry}
\usepackage{ctable}
\usepackage{array}
\usepackage{graphicx}
\usepackage{fancyvrb}

% Paragraph spacing
\setlength{\parindent}{0em}
\setlength{\parskip}{0.5em}

% Default font
\renewcommand*{\familydefault}{\sfdefault}

% table lines
\newcolumntype{?}{!{\vrule width 1pt}}

% hyperlinks
\hypersetup{
  breaklinks=true,  % so long urls are correctly broken across lines
  colorlinks=true,
  urlcolor=blue,
  linkcolor=red,
  citecolor=red,
 }

\begin{document}

% Header info
\textbf{CSE 291 - PERSONAL GENOMICS FOR BIOINFORMATICIANS - PROJECT SUGGESTIONS}

\section*{Project suggestions}

\begin{itemize}
	\item Comparing existing tools
	\begin{itemize}
		\item Explore risk prediction tools/methods, and the effect of various methods for controlling for ancestry and linkage disequilibrium.
		\item Evaluate and compare different methods for annotating non-coding variants. 
	\end{itemize}
	\item Prototype a novel bioinformatics tool
	\begin{itemize}
		\item Genotyping repeats (e.g. STRs, VNTRs, Alu/Line) from long-read or synthetic long read technologies (10X, Nanopore)
		\item Detecting mosaicism at complex variants (e.g. STRs) from next-generation sequencing data
	\end{itemize}
	\item A visualization tool for making genomic data more accessible to non-experts
	\begin{itemize}
		\item Visualizing annotations from VCF files (e.g. see \href{http://vcf.iobio.io/}{http://vcf.iobio.io/} for inspiration)
		\item Choose a trait not already presented by 23andMe or DNA.land, and create a web page predicting a person's trait/risk for a complex trait that provides a tutorial behind the genetics of the trait and a visualization of the contribution of different SNPs.
		\item A web page for use in forensics, that predicts relevant traits (e.g. ancestry, eye color, skin color, or even height and more complex phenotypes) from a genotype file, along with explanations for lay people.
	\end{itemize}
	\item Novel analysis of genomic data
	\begin{itemize}
		\item (idea from Stanford course): tool to simulate "offspring" from a potential "mating" between two genomes. Analyze distribution of traits in siblings.
		\item Explore importance of non-coding annotations using variance partitioning.
	\end{itemize}
\end{itemize}

\section*{Public datasets}

The following are public datasets that are available for the project:
\begin{itemize}
	\item The 1000 Genomes Project \href{http://www.internationalgenome.org/}{http://www.internationalgenome.org/} Sequencing data and variant calls for about 3,000 samples worldwide
	\item the Simons Genome Diversity Project \href{https://www.simonsfoundation.org/life-sciences/simons-genome-diversity-project-dataset/}{https://www.simonsfoundation.org/life-sciences/simons-genome-diversity-project-dataset/} 300 high coverage whole genomes from diverse population groups. 
	\item The ENCODE Project \href{https://www.genome.gov/10005107/}{https://www.genome.gov/10005107/} Histone modifications, transcription factor, expression, and more data from a variety of human cell types.
	\item The GTEx Project \href{gtexportal.org}{gtexportal.org}. Expression data and QTLs from hundreds of human cell types.
	\item Familinx \href{http://www.familinx.org/}{http://www.familinx.org/}. Pedigree info for millions of individuals.
\end{itemize}

\end{document}