\documentclass[12pt]{article}

% Imports
\usepackage{hyperref}
\usepackage[margin=0.5in]{geometry}
\usepackage{ctable}
\usepackage{array}
\usepackage{graphicx}
\usepackage{fancyvrb}

% Paragraph spacing
\setlength{\parindent}{0em}
\setlength{\parskip}{0.5em}

% Default font
\renewcommand*{\familydefault}{\sfdefault}

% table lines
\newcolumntype{?}{!{\vrule width 1pt}}

% hyperlinks
\hypersetup{
  breaklinks=true,  % so long urls are correctly broken across lines
  colorlinks=true,
  urlcolor=blue,
  linkcolor=red,
  citecolor=red,
 }

\begin{document}

% Header info
\textbf{CS 291 - PERSONAL GENOMICS FOR BIOINFORMATICIANS}

\section*{Problem Set 2 - Ancestry}

All homework should be sent to mgymrek@ucsd.edu with subject line \textcolor{purple}{[CSE291 PS2:LASTNAME]} by the beginning of class on \textbf{Tuesday, January 24}. % TODO where to find data/tools

\subsection*{Objectives}
\begin{itemize}
\item Determine ancestry in homogeneous and admixed genomes.
\item Predict relationships between related samples.
\item Impute missing variants using imputation from a reference panel.
\end{itemize}

\subsection*{Analyzing ancestry using principal components analysis}
% Perform PCA on 1000 Genomes and plot first 2 PCs (with 3 samples held out), single chromosome
% Question: 3 major clusters?
% Question: what does 1st PC describe? 2nd PC?
% Project 3 samples onto the PCs. What population do they fall in?
% Run time with subset of samples, how does it scale?
% Real world tools: smartpca, fastpca

% TODO data:
% 1KG VCF for chr16, minus 3 samples
% List of samples/pop/color for held in samples
% VCF and list of held out samples in separate file
% Optional python skeleton 
% - get vcf into matrix, normalize snps, perform PCA, plot first 2 Pcs colored
% - option for number of samples to use, number of snps to use

% NA10847 CEU
% NA18923 YRI
% NA19700 ASW

\subsection*{Chromosome painting}

\subsection*{Relative finding}

\subsection*{Imputing missing variants}

\end{document}